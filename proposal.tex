\documentclass[12pt]{article}

%% preamble: Keep it clean; only include those you need
\usepackage{amsmath}
\usepackage[margin = 1in]{geometry}
\usepackage{graphicx}
\usepackage{booktabs}
\usepackage{natbib}

% for space filling
\usepackage{lipsum}
% highlighting hyper links
\usepackage[colorlinks=true, citecolor=blue]{hyperref}


%% meta data

\title{Proposal: Effect of Native Climate on Quarterback Performance in the NFL}
\author{Lydia Margolien\\
  Department of Statistics\\
  University of Connecticut
}

\begin{document}
\maketitle


\paragraph{Introduction}

Player performance in extreme temperature and weather conditions is often a topic of 
conversation for the both the casual sports fan and the professional analyst. Within 
the American National Football League (NFL), the season begins with very warm temperatures
and can dip into negative degrees Farenheit in the January and February months.  Most 
locales in the United States experience extreme heat at least a few time each season, 
thus acclimating even northern players to the scorching sun and high temperatures that 
may be common in the southern United States. However, many southern states do not experience 
the same freezing temperatures as most of the mid-Atlantic, northeastern, and midwestern 
regions do. The NFL plays outdoors in the United States from the beginning of September through
February and experiences seasonal shifts into summer, autumn, and winter. 

\lipsum[1] 

\paragraph{Specific Aims}
This paper seeks to test if there is a relationship between the climate of a player's hometown 
and how they perform in colder temperatures. This paper examines quarterback performance in three 
areas: Yards per attempt (Yards), Completion Percentage (Comp), and Completion Percentage over 
Expectation (COMPE). Specifically, this paper investigates whether players from warmer climates 
perform worse in the cold compared to their counterparts who grew up in colder climates. 

\lipsum[2]

\paragraph{Data}
Data will be used from the 2017-2018, 2018-2019, and 2019-2020 seasons. Quarterbacks who grew up
in states that rarely experienced freezing temperatures will be evaluated on their Yards, Comp, 
and COMPE statistics for games played in freezing temperatures (below 32 degrees Farenheit). 
Player data will be taken from the ESPN player performance webpages. Climate and weather averages 
for stadium location and player home town will be taken from climate.gov. 

\lipsum[3]

\paragraph{Research Design and Methods}
Players will be selected who were raised in, and played college football, in states with monthly 
temperature averages greater than 50 degrees Farenheit in January and Feburary, the two coldest 
months of the year in the Northern hemisphere. A two-sample Z-test will be run to compare if 
there is evidence to suggest that warmer climate causes lower performance in freezing temperatures. 

\lipsum[4]

\paragraph{Discussion}
Results of this paper could be used to predict player performance in certain games, especially 
in the post-season when temperatures are at their lowest and the stakes are at their highest. 
Results of this paper could also encourage further investigation into other performance metrics 
for other player positions. For instance, one could see if there is an effect of native player 
climate on quarterback performance but not on running back performance.

\lipsum[5] 

\paragraph{Conclusion}
\lipsum[1]


\bibliography{../manuscript/refs}
\bibliographystyle{chicago}

\end{document}
